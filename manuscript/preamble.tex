\IEEEoverridecommandlockouts
\usepackage{cite}
\usepackage{amsmath,amssymb,amsfonts}
\usepackage{algorithmic}
\usepackage{graphicx}
\usepackage{textcomp}
\usepackage{xcolor}
\usepackage{tabularx}
\usepackage{tikz}
\usepackage{url}

\usepackage{cuted}
\usetikzlibrary{arrows.meta,positioning,calc}
\usepackage{amsthm}
\usepackage{amsmath,amsthm,amsfonts,amssymb,amscd}
\theoremstyle{definition}
\newtheorem{definition}{Definition}[section]
% User prefers \begin{defintion} ... \end{defintion}
\newtheorem{defintion}[definition]{Definition}
\newtheorem{theorem}{Theorem}[section]
\newtheorem{lemma}[theorem]{Lemma}
\newtheorem{corollary}[theorem]{Corollary}
\newtheorem{proposition}[theorem]{Proposition}
\newtheorem{remark}{Remark}[section]
\newtheorem{example}{Example}[section]
\newtheorem{assumption}{Assumption}[section]
\def\BibTeX{{\rm B\kern-.05em{\sc i\kern-.025em b}\kern-.08em
    T\kern-.1667em\lower.7ex\hbox{E}\kern-.125emX}}
% Short math macros
\newcommand{\FF}{\mathcal{F}}
\newcommand{\IFF}{\mathcal{F}^{-1}}
\newcommand{\R}{\mathbb{R}}
\newcommand{\E}{\mathbb{E}}
\usepackage{algorithm,algorithmic}

\usepackage{fancyhdr}
