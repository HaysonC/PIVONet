\section{Implementation and Reproducibility} \label{apx:implementation}
All code lives in \texttt{\url{github.com/haysonc/pivonet}}. Clone the repository, install dependencies, and run the VSDE model with the following steps:
\begin{enumerate}
\item Set up the virtual environment and install the requirements:
\begin{verbatim}
python -m venv venv
source venv/bin/activate
pip install -r requirements.txt
\end{verbatim}
\item Use the \texttt{pivo} CLI to orchestrate training/inference or call the flow module:\\
\begin{verbatim}
pivo
\end{verbatim}
\end{enumerate}
Dataset from \texttt{PyFR} is uploaded on Google Drive: \texttt{\url{https://drive.google.com/file/d/1MWayzNu9oW745n9ZPRmS7aWnw8wthX8k/view?usp=drive_link}}. You can download and extract it into the \texttt{data/} folder in the repository root.

A sample repository stucture is as follows:
\begin{verbatim}
pivonet/
|-- data/
|   |-- 2d-euler-vortex/
|   `-- ...
|-- src/
|   |-- models/
|   |-- utils/
|   `-- ...
`-- ...
\end{verbatim}
For more detailed instructions, refer to the \texttt{README.md} in the repository root.
The code style is enforced with \texttt{ruff} and is configured via \texttt{pyproject.toml} ; all configuration values follow YAML-like semantics with leading keys for clarity, so keep any new scripts aligned with those defaults.
