\section{Results}

This section presents experimental results corresponding to the hypotheses outlined in Section IV-A. We evaluate the proposed CNF–VSDE framework in terms of model expressiveness, the contribution of variational stochastic control, generalization across distinct flow regimes, and numerical stability under different ODE integration schemes.

Model performance is primarily quantified using mean absolute error (MAE) across all evaluated scenarios. MAE is selected due to its interpretability, as it measures average prediction error in the same physical units as the target variables, facilitating direct comparison with ground-truth fluid trajectories. In addition, MAE provides a linear error measure that does not disproportionately penalize large deviations, which is desirable in fluid dynamics applications where consistent accuracy is preferred over sensitivity to rare extreme events. Compared to squared-error metrics, MAE also offers increased robustness to occasional anomalies arising from turbulence or numerical noise.
\subsection{Expressiveness of ODE Backbones for Fluid Dynamics}
We first evaluate the performance of different ODE-based backbone models in capturing fluidic dynamics. Overall, the ODE-only approach is able to learn the dominant, mean flow behavior and produces smooth, physically consistent trajectories that align with the large-scale structure of the flow. In particular, the homogeneous CNF successfully converges to a stable solution that reflects the primary deterministic dynamics of the incompressible cylinder flow.

However, this same determinism becomes a limitation when modeling stochastic fluid behavior. Although the model implicitly learns a probability distribution over trajectories, the absence of latent variables restricts its expressive capacity. As a result, the learned dynamics collapse toward a single dominant mode, making the model overly conservative and preventing it from capturing the full range of stochastic variations present in the system. This leads to an underrepresentation of flow variability and fine-scale randomness observed in the ground truth data. The inference results of the ODE-only model are shown in Figure~\ref{fig:inc-cylinder-ode}.

\begin{figure}[h!]
\centering
\includegraphics[width=1\linewidth]{figures/results/inc-cylinder/ODE_inference.png}
\caption{Inference results of the ODE-only CNF model on the incompressible cylinder flow. Predicted trajectories are overlaid on the ground truth flow realizations. While the model accurately captures the overall mean flow structure and produces smooth, stable trajectories, the predictions collapse toward a deterministic solution. This behavior highlights the model’s inability to represent stochastic variations and multi-modal flow behavior, resulting in limited diversity compared to the ground truth dynamics.}
\label{fig:inc-cylinder-ode}
\end{figure}

Additionally, we can copare the integrator performance using different numerical methods. We observed no significant difference in model accuracy or stability when using Euler, RK4, or Dormand-Prince integrators for the ODE backbone. This suggests that the learned dynamics are sufficiently smooth and well-behaved, allowing lower-order methods like Euler to perform comparably to higher-order schemes. The choice of integrator thus has minimal impact on the model’s ability to capture fluid behavior in this scenario. A comparison of integrator performance is shown in Appendix~\ref{apx:figures}, Figure~\ref{fig:integrator_comparison}.

\subsection{Impact of Variational Stochastic Control on Trajectory Fidelity}

To improve alignment with physical variability, we evaluate the effect of incorporating variational stochastic differential equation (VSDE) control into the CNF backbone. This analysis directly addresses whether stochastic augmentation enhances trajectory fidelity and uncertainty representation.

\begin{figure}[h!]
    \centering
    \includegraphics[width=1\linewidth]{figures/results/shock-tube/vsde_vs_cnf_difference.png}
    \caption{Regional comparison of mean absolute error (MAE) between the CNF-only and CNF–VSDE models, showing consistently lower error for VSDE across all spatial regions.}
    \label{fig:regional_error}
\end{figure}

Fig.~\ref{fig:regional_error} presents a spatial comparison of MAE between the CNF-only and CNF–VSDE models. Across all spatial regions, the VSDE-enhanced model consistently achieves lower reconstruction error. These improvements indicate that the learned posterior control actively compensates for deviations introduced by the frozen CNF drift, particularly in regions where stochastic effects play a significant role.


\begin{figure}[h!]
    \centering
    \includegraphics[width=1\linewidth]{figures/results/shock-tube/timestep_mae.png}
    \caption{Comparison of ensemble trajectory behavior between the CNF-only and VSDE models. The CNF-only model produces relatively deterministic trajectories, whereas the VSDE framework generates ensembles exhibiting realistic dispersion. Although the VSDE model initially struggled to capture the accurate representation of turbulent and shear-dominated regions, the trajectories match the stochastic behavior observed in the CFD ground truth.}
    \label{fig:timestep_mae}
\end{figure}

\begin{figure}[h!]
    \centering
    \includegraphics[width=1\linewidth]{figures/results/distribtution.png}
    \caption{Distribution of final-position MAE for the viscous shock tube. VSDE control shifts the error distribution toward lower values relative to the CNF-only model, indicating improved trajectory fidelity without sacrificing ensemble diversity.}
    \label{fig:mae_error}
\end{figure}


\subsection{Physical Consistency in Velocity-Space Dynamics}

Physical consistency is further evaluated by comparing velocity phase-space structure and speed distributions against CFD reference statistics. As shown in Fig.~10, the ground-truth velocity field exhibits pronounced anisotropy in $(v_x, v_y)$ space and a broad, non-Gaussian speed distribution characteristic of shear- and turbulence-driven flow at early times. 

The CNF–VSDE model successfully reproduces these velocity-space patterns and preserves the empirical spread of particle speeds. In contrast, the CNF-only model collapses trajectories into a narrower region of velocity space, indicating an underrepresentation of physical variability. These results suggest that variational stochastic control enables the model to capture physically meaningful velocity distributions rather than merely approximating mean transport behavior.

A plot of generated trajectories overlays is included in Appendi~\ref{apx:figures}, Figure~\ref{fig:trajectory_overlay}.

